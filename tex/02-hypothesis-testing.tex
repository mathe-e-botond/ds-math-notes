\chapter{Hypothesis testing}

Hypothesis testing is the process to confirm a test metric on a data set. The general process is to

\begin{enumerate}
    \item State a \textbf{null hypothesis\index{null hypothesis}} \(H_0\) which is the contradiction of the \textbf{alternative hypothesis\index{alternative hypothesis}} we want to verify, sometimes noted with \(H_1\)
    \item Use a \textbf{test statistic\index{test statistic}} to calculate the probability of an observation given the null hypothesis. This probability is the \textbf{p-value\index{p-value}}
    \item Compare the p-value to a target \(\alpha\) \textbf{significance level\index{significance level}}
\end{enumerate}

We say a null hypothesis is one tailed if

\[H_0: \mu = \mu_0,\ H_1 : \mu > \mu_0 \text{ or } H_1 : \mu < \mu_0\]

a two-tailed test is

\[H_0: \mu = \mu_0,\ H_1 : \mu \ne \mu_0\]

for some metric \(\mu\)

\section{Z-test and t-test}

\section{ANOVA}

ANOVA is used to verify means of multiple populations. If we apply Z-test multiple times, the error accumulates.

The ANOVA Hypothesis for \(p\) groups:

\[
\begin{aligned}
&H_0: \mu_1 = \mu_2 = ... = \mu_p,\  \\
&H_1: \mu_i \ne \mu_j,\ \forall i, j \in \{1, ...\ , p\}
\end{aligned}
\]


